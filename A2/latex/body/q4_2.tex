\textbf{4.2}. Implement linear interpolation smoothing between unigram, bigram, and trigram models: 
    $$\theta_{x_j \vert x_{j-2}, \, x_{j-1}}' = \lambda_1 \theta_{x_j} + \lambda_2 \theta_{x_j \vert x_{j-1}} + \lambda_3 \theta_{x_j \vert x_{j-2}, \, x_{j-1}}$$ 
Where $\theta'$ represents the smoothed parameters and the hyperparameters $\lambda_1$, $\lambda_2$, and $\lambda_3$ are weights on the unigram, bigram, and trigram language models, respectively. $\lambda_1 + \lambda_2 + \lambda_3 = 1$. Provide graphs, tables, charts, or other summary evidence to support any claims you make.

\begin{enumerate}
    \item Describe your models and experimental procedure.
	\begin{quote}
    \textbf{Linear Interpolation n-Gram Model}
\end{quote}

\begin{itemize}
    \item Given an $n>1$, build n k-Gram models where $1\leq k \leq n$ such as those described above.
    \item For each token $t_i$ in the nth k-Gram model compute the linear interpolation smoothing using the formula described above.
\end{itemize}


    \item Report perplexity scores on training, validation, test sets for various values of $\lambda_1$, $\lambda_2$, and $\lambda_3$. Report no more than 5 different sets of $\lambda$’s. In addition to this, report the training and validation perplexity for the values $\lambda_1 = 0.1$, $\lambda_2 = 0.3$, and $\lambda_3 = 0.6$
	\begin{quote}
    Perplexity and Percentage of Tokens Removed by Linear Interpolation Trigram models with different Lambda Parameters on Training, Validation, and Testing datasets with an unk threshold of 3.
\end{quote}
\begin{center}
    \begin{tabular}{ |c|c|c|c| } 
	\hline
	$\lambda_1$, $\lambda_2$, $\lambda_3$ & Training & Validation & Testing \\ 
	\hline
	0.0, 0.1, 0.9 & 7.82 & 43.79 & 43.58 \\ 
	\hline
	0.1, 0.3, 0.6 & 10.40 & 289.71 & 288.47 \\ 
	\hline
	0.3, 0.4, 0.3 & 16.94 & 230.65 & 230.19 \\ 
	\hline
	0.6, 0.3, 0.1 & 36.22 & 257.18 & 257.30 \\ 
	\hline
	0.9, 0.1, 0.0 & 282.09 & 437.69 & 440.06 \\ 
	\hline
    \end{tabular}
\end{center}
\begin{quote}
    The trend is that {\bf increasing $\lambda_3$ decreases perplexity} while {\bf increasing $\lambda_1$ increases perplexity}.
\end{quote}

    \item If you use half of the training data, would it increase or decrease the perplexity on previously unseen data? Why? Provide empirical experimental evidence if necessary.
	\begin{quote}
    Perplexity and Percentage of Tokens removed by n-Gram models after training on half the training dataset on Training, Validation, and Testing datasets with an unk threshold of 3.
\end{quote}

\begin{center}
    \begin{tabular}{ |c|c|c|c|c|c|c| } 
	\hline
	 & \multicolumn{2}{c|}{Training} & \multicolumn{2}{c|}{Validation} & \multicolumn{2}{c|}{Testing} \\ 
	\hline
	Model & Perplexity & \% Removed & Perplexity & \% Removed & Perplexity & \% Removed \\
	\hline
	Unigram & 816.49 & 0.00 & 723.12 & 0.00 & 725.55 & 0.00 \\
	\hline
	Bigram & 63.08 & 0.00 & 52.09 & 22.07 & 46.24 & 19.71 \\ 
	\hline
	Trigram & 6.11 & 0.00 & 9.56 & 58.52 & 8.20 & 55.84 \\
	\hline
    \end{tabular}
\end{center}

\begin{quote}
    Perplexity and Percentage Removed percentage differences from training model on a half training dataset from training model on the entire training dataset. Positive values indicate an increase in perplexity from entire training dataset to half training dataset and negative values indicate a decrease in perplexity from entire training dataset to half training dataset.
\end{quote}

\begin{center}
    \begin{tabular}{ |c|c|c|c|c|c|c| } 
	\hline
	 & \multicolumn{2}{c|}{Training} & \multicolumn{2}{c|}{Validation} & \multicolumn{2}{c|}{Testing} \\ 
	\hline
	Model & Perplexity & \% Removed & Perplexity & \% Removed & Perplexity & \% Removed \\
	\hline
	Unigram & -16.39 & 0.00 & -12.80 & 0.00 & -19.07 & 0.00 \\
	\hline
	Bigram & -18.15 & 0.00 & -20.56 & 9.75 & -24.81 & 4.67 \\ 
	\hline
	Trigram & -16.30 & 0.00 & -14.95 & 3.89 & -20.31 & 2.12 \\ 
	\hline
    \end{tabular}
\end{center}


\begin{quote}
    Compared to training the same model son the entire training dataset, {\bf reducing the size of the training dataset} has the effect of {\bf reducing the perplexity} across all models and datasets. However this also has the effect of {\bf increasing the percentage of words removed} since fewer words had been seen during training. Pushed to extremes, we expect that a training dataset of size 0 would therefore remove 100\% of words. In conclusion, a balance should be sought between dataset size, percentage of words removed, and perplexity.
\end{quote}

    \item If you convert all tokens that appeared less than 5 times to $<$UNK$>$, would it increase or decrease the perplexity on the previously unseen data compared to an approach that converts only a fraction of words that appeared just once to $<$UNK$>$? Why? Provide empirical experimental evidence if necessary.
	\begin{quote}
    Perplexity and Percentage of Tokens removed by n-Gram models on Training, Validation, and Testing datasets with an unk threshold of 5.
\end{quote}

\begin{center}
    \begin{tabular}{ |c|c|c|c|c|c|c| } 
	\hline
	 & \multicolumn{2}{c|}{Training} & \multicolumn{2}{c|}{Validation} & \multicolumn{2}{c|}{Testing} \\ 
	\hline
	Model & Perplexity & \% Removed & Perplexity & \% Removed & Perplexity & \% Removed \\
	\hline
	Unigram & 803.49 & 0.00 & 754.30 & 0.00 & 756.69 & 0.00 \\ 
	\hline
	Bigram & 75.63 & 0.00 & 64.23 & 18.00 & 60.57 & 16.84 \\ 
	\hline
	Trigram & 7.95 & 0.00 & 11.61 & 53.74 & 10.67 & 52.08 \\ 
	\hline
    \end{tabular}
\end{center}

\begin{quote}
    Perplexity percentage difference of training using an unk threshold of 3 compared to using an unk threshold of 5. Positive values indicate an increase in perplexity from unk threshold of 3 to unk threshold of 5 and negative values indicate a decrease in perplexity from an unk threshold of 3 to an unk threshold of 5.
\end{quote}

\begin{center}
     \begin{tabular}{ |c|c|c|c|c|c|c| } 
	\hline
	 & \multicolumn{2}{c|}{Training} & \multicolumn{2}{c|}{Validation} & \multicolumn{2}{c|}{Testing} \\ 
	\hline
	Model & Perplexity & \% Removed & Perplexity & \% Removed & Perplexity & \% Removed \\
	\hline
	Unigram & -19.44 & 0.00 & -16.76 & 0.00 & -16.94 & 0.00 \\ 
	\hline
	Bigram & -1.86 & 0.00 & -2.06 & -11.02 & -1.52 & -11.16 \\ 
	\hline
	Trigram & 8.52 & 0.00 & 3.24 & -4.71 & 3.63 & -4.87 \\ 
	\hline
    \end{tabular}
\end{center}

\begin{quote}
Compared to using an unk threshold of 3, {\bf increasing the unk threshold to 5} has the effect of {\bf decreasing} perplexity on {\bf unigram and bigram models} and of {\bf increasing} perplexity on the {\bf trigram model} and of {\bf decreasing} the percentage of words removed across all models. This suggests that there exists a balance to strike between n and the size of unk threshold to minimize perplexity and the percentage of words removed.  
\end{quote}

\end{enumerate}
