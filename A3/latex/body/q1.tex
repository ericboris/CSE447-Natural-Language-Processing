{\bf 1.} 
A simple way to compute a vector representation of a sequence of words is to add up the vector representa- tions of the words in the sequence. Consider a sentiment analysis model in which the predicted sentiment is given by:

\begin{equation}
    \textrm{score}(w_1, ..., w_m) = \mathbb{\theta} \cdot \sum_{i=1}^{m} \mathbf{x}_{w_i}
\end{equation}

where $w_i$ is the $i$th word and $\mathbf{x}_{w_i}$ is the embedding for the $i$th word; the input is of length $m$ (in word tokens).
$\mathbb{\theta}$ are parameters. 
Prove that, in such a model, the following two inequalities cannot both hold:

\begin{align}
    \textrm{score(good)} &> \textrm{score(not good)} \\
    \textrm{score(bad)} &< \textrm{score(not bad)}
\end{align}

\begin{quote}
{\bf Answer}: Prove by contradiction that inequalites (2) and (3) cannot both be true when using sentiment analysis model (1). \\

    Assume that both inequalities (2) and (3) are true using sentiment analysis model (1).
    Let $\textbf{x}_{w_1}$, $\textbf{x}_{w_2}$, $\textbf{x}_{w_3}$ be arbitrary word embedding vectors for the words ``good'', ``not'', and ``bad'', respectively and let $\mathbf{\theta}$ be an arbitrary weight vector. \\

    {\bf Claim 1}: $0 > \mathbf{\theta} \cdot \textbf{x}_{w_2}$
    \begin{align*}
	\textrm{score(good)} &> \textrm{score(not good)} \\
	\mathbf{\theta} \cdot \sum_{i=1}^{m} \textbf{x}_{w_i} &> \mathbf{\theta} \cdot \sum_{i=1}^{m} \textbf{x}_{w_i} \\
	\mathbf{\theta} \cdot \textbf{x}_{w_1} &> \mathbf{\theta} \cdot \left( \textbf{x}_{w_2} + \textbf{x}_{w_1} \right) \\
	\mathbf{\theta} \cdot \textbf{x}_{w_1} &> \mathbf{\theta} \cdot \textbf{x}_{w_2} + \mathbf{\theta} \cdot \textbf{x}_{w_1} \\
	0 &> \mathbf{\theta} \cdot \textbf{x}_{w_2}
     \end{align*}
    {\bf Claim 2}: $0 < \mathbf{\theta} \cdot \textbf{x}_{w_2}$
    \begin{align*}
	\textrm{score(bad)} &< \textrm{score(not bad)} \\
	\mathbf{\theta} \cdot \sum_{i=1}^{m} \textbf{x}_{w_i} &< \mathbf{\theta} \cdot \sum_{i=1}^{m} \textbf{x}_{w_i} \\
	\mathbf{\theta} \cdot \textbf{x}_{w_3} &< \mathbf{\theta} \cdot \left( \textbf{x}_{w_2} + \textbf{x}_{w_3} \right) \\
	\mathbf{\theta} \cdot \textbf{x}_{w_3} &< \mathbf{\theta} \cdot \textbf{x}_{w_2} + \mathbf{\theta} \cdot \textbf{x}_{w_3} \\
	0 &< \mathbf{\theta} \cdot \textbf{x}_{w_2}
    \end{align*}
    By {\bf Claim 1} and {\bf Claim 2}, $0 > \mathbf{\theta} \textbf{x}_{w_2}$ and $0 < \mathbf{\theta} \textbf{x}_{w_2}$, which is a contradiction. Thus, inequalities (2) and (3) cannot both be true when using sentiment analysis model (1).
\end{quote}


Next, consider a slightly different model:

\begin{equation}
    \textrm{score}(w_1, ..., w_m) = \frac{1}{m} \left( \mathbb{\theta} \cdot \sum_{i=1}^{m} \mathbf{x}_{w_i} \right)
\end{equation}

Construct an example of a pair of inequalities similar to (2–3) that cannot both hold.

\begin{quote}
    {\bf Answer}: Prove by contradiction that the following inequalities (5) and (6) cannot both hold when using sentiment analysis model (4).
    \begin{align}
	\textrm{score(good good)} &> \textrm{score(not good)} \\
	\textrm{score(good bad)} &< \textrm{score(not bad)}
    \end{align}
    Assume that both inequalities (5) and (6) are true using sentiment analysis model (4). Let $\textbf{x}_{w_1}$, $\textbf{x}_{w_2}$, $\textbf{x}_{w_3}$ be arbitrary word embedding vectors for the words ``good'', ``not'', and ``bad'', respectively and let $\mathbf{\theta}$ be an arbitrary weight vector. \\ 

    {\bf Claim 1}: $\mathbf{\theta} \cdot \textbf{x}_{w_1} > \mathbf{\theta} \cdot \textbf{x}_{w_2}$
    \begin{align*}
	\textrm{score(good good)} &> \textrm{score(not good)} \\
	\frac{1}{m} \left( \mathbf{\theta} \cdot \sum_{i=1}^m \textbf{x}_{w_i} \right) &> \frac{1}{m} \left( \mathbf{\theta} \cdot \sum_{i=1}^m \textbf{x}_{w_i} \right) \\
	\frac{1}{2} \left( \mathbf{\theta} \cdot \left( \textbf{x}_{w_1} + \textbf{x}_{w_1} \right) \right) &> \frac{1}{2} \left( \mathbf{\theta} \cdot \left( \textbf{x}_{w_2} + \textbf{x}_{w_1} \right) \right) \\
    \mathbf{\theta} \cdot \left( \textbf{x}_{w_1} + \textbf{x}_{w_1} \right) &> \mathbf{\theta} \cdot \left( \textbf{x}_{w_2} + \textbf{x}_{w_1} \right) \\
	\mathbf{\theta} \cdot \textbf{x}_{w_1} + \mathbf{\theta} \cdot \textbf{x}_{w_1}  &> \mathbf{\theta} \cdot \textbf{x}_{w_2} + \mathbf{\theta} \cdot \textbf{x}_{w_1} \\
	\mathbf{\theta} \cdot \textbf{x}_{w_1} &> \mathbf{\theta} \cdot \textbf{x}_{w_2}
    \end{align*}

    {\bf Claim 2}: $\mathbf{\theta} \cdot \textbf{x}_{w_1} < \mathbf{\theta} \cdot \textbf{x}_{w_2}$
    \begin{align*}
	\textrm{score(good bad)} &< \textrm{score(not bad)} \\
	\frac{1}{m} \left( \mathbf{\theta} \cdot \sum_{i=1}^m \textbf{x}_{w_i} \right) &< \frac{1}{m} \left( \mathbf{\theta} \cdot \sum_{i=1}^m \textbf{x}_{w_i} \right) \\
	\frac{1}{2} \left( \mathbf{\theta} \cdot \left( \textbf{x}_{w_1} + \textbf{x}_{w_3} \right) \right) &< \frac{1}{2} \left( \mathbf{\theta} \cdot \left( \textbf{x}_{w_2} + \textbf{x}_{w_3} \right) \right) \\
    \mathbf{\theta} \cdot \left( \textbf{x}_{w_1} + \textbf{x}_{w_3} \right) &< \mathbf{\theta} \cdot \left( \textbf{x}_{w_2} + \textbf{x}_{w_3} \right) \\
	\mathbf{\theta} \cdot \textbf{x}_{w_1} + \mathbf{\theta} \cdot \textbf{x}_{w_3}  &< \mathbf{\theta} \cdot \textbf{x}_{w_2} + \mathbf{\theta} \cdot \textbf{x}_{w_3} \\
	\mathbf{\theta} \cdot \textbf{x}_{w_1} &< \mathbf{\theta} \cdot \textbf{x}_{w_2}
    \end{align*}

    By {\bf Claim 1} and {\bf Claim 2}, $\mathbf{\theta} \cdot \textbf{x}_{w_1} > \mathbf{\theta} \cdot \textbf{x}_{w_2}$ and $\mathbf{\theta} \cdot \textbf{x}_{w_1} < \mathbf{\theta} \cdot \textbf{x}_{w_2}$, which is a contradiction. Thus, inequalities (5) and (6) cannot both be true when using sentiment analysis model (4).
\end{quote}

